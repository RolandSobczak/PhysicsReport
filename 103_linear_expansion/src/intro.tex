\section{Introduction}

The aim of this experiment is to investigate the phenomenon of linear thermal expansion of solid materials and to determine the linear expansion coefficient of the tested materials.

Thermal expansion is the change in the dimensions of a body due to a change in temperature. For small temperature changes, the length of a body changes proportionally to the temperature difference, according to the relation:
\begin{equation}
    \Delta L = \alpha \, L_0 \, \Delta T
\end{equation}
where:
\begin{itemize}
    \item $\Delta L$ – change in length,
    \item $L_0$ – initial length of the body,
    \item $\Delta T$ – temperature change,
    \item $\alpha$ – linear expansion coefficient.
\end{itemize}

This experiment allows us to verify this relationship and determine the value of $\alpha$ for metals, which is important when designing structures exposed to temperature variations.

